As with any RAG based solution, a vector database is a critical component of the architecture.
It serves the purpose of the main \emph{knowledge base} for the system.

To simplify the setup process, the document ingestion component and chat completion component are built into unified micro-service.
This is written in \emph{Java} using \emph{Spring Framework} as it was deemed comfortable with team expertise of the members.

Following are the steps to setup the microservice locally.


Clone the sources from GitHub Repository:
\begin{lstlisting}
 git clone git@github.com:AyeshJayasekara/GovHack-2025-Backend.git
\end{lstlisting}

Build the executable from sources (Execute following within the cloned directory that contains the \emph{pom.xml} file):
\begin{lstlisting}
 mvn clean install -DskipTests
\end{lstlisting}

To spin up the service from the JAR file generated:
\begin{lstlisting}
 cd target
 java -jar service-1.0.0.jar
\end{lstlisting}

At this point, the service should spin up. It is important to note that the Mistral service is required at the startup.
If any problems persist, repeat instructions above and make sure that service is healthy before attempting to troubleshoot.


\newsection{Populating the Vectorised Knowladge Base}

At this point the local setup is ready to be populated with metadata of the system.
To simplify this process a meta-data file has been prepared for following data sources as outlined in the challenge.

\begin{itemize}
    \item \textbf{Data Set 1:} \href{https://www.data.gov.au/data/dataset/freedom-of-information-statistics}{Freedom of Information Statistics on Data.gov.au}
    \item \textbf{Data Set 2:} \href{https://www.kaggle.com/datasets/manishkumar21324/employee-leave-tracking-data}{Employee Leave Tracking Data on Kaggle}
\end{itemize}

See \href{https://github.com/AyeshJayasekara/GovHack-2025-Backend/tree/main/DATA}{GovHack 2025 Backend - DATA Directory} for metadata we
prepared as part of data preparation step.

Once you are ready execute below to instruct the backend service to populate the knowledge base.

\begin{lstlisting}
curl --location --request POST 'localhost:8080/rag/ingest'
\end{lstlisting}


\textbf{Congratulations! You are now ready to explore the capabilities of the PoC system.}

\clearpage

If you do not want to set up the front-end application to test the system in user-friendly way,
you can interact with the system from terminal in the form of HTTP requests.

Use the following cURL code snipped as a template and alter the \textbf{\emph{question}} parameter with your desired prompt to retrieve a response from the system.
As to be expected, the terminal view may not be most reader friendly so we recommend to use the provided frontend system.

\begin{lstlisting}
curl --location 'localhost:8080/ask' \
--header 'Content-Type: application/json' \
--data '{
    "question": "What is happening with leave patterns in my team?"
}'\end{lstlisting}
